\documentclass[11pt]{article}
\usepackage{amssymb}
\usepackage{amsthm}
\usepackage{enumitem}
\usepackage{amsmath}
\usepackage{bm}
\usepackage{adjustbox}

\title{\textbf{Solved Abstract Algebra - Pinter}}
\author{Franco Zacco}
\date{}

\addtolength{\topmargin}{-3cm}
\addtolength{\textheight}{3cm}

\begin{document}

\maketitle
\thispagestyle{empty}

\section*{Chapter 3 - The definition of groups}

	\subsubsection*{Problem A}

		Prove that each of the following sets, with the indicated operation, is an Abelian group
		\begin{itemize}
			\item [\textbf{1}] $x * y = x+y+k$ ($k$ a fixed constant), on the set $\mathbb{R}$.
			\item [\textbf{2}] $x*y=\frac{xy}{2}$, on the set $\{ x \in \mathbb{R}: x \neq 0\}$
		\end{itemize}

	\subsubsection*{Solutions}

		\begin{proof}{\textbf{1}}
		First, it's necessary to prove that this set with the indicated operation is a group:
		\begin{itemize}
			\item [(i)]			
			$x*(y*z)=x+(y+z+k)+k = x+y+z+2k$ \\
			$(x*y)*z=(x+y+k)+z+k = x+y+k+z+k=x+y+z+2k$ \\
			Therefore $*$ is an associative operation.
			\item [(ii)] $x*e=x$ then $x+e+k=x \Rightarrow e=-k$
			\item [(iii)] $x*x'=e$ then $x+x'+k=e \Rightarrow x'=-x-2k$
			\item [(iv)] Finally, for this set with this operation to be an Abelian group, it's necessary that the operation be commutative, so \\
			$x*y =x+y+k$ and $y*x=y+x+k=x+y+k$.\\
			Therefore $*$ is a commutative operation and with the set $\mathbb{R}$, they form an Abelian group.
		\end{itemize}
		\end{proof}

		\begin{proof}{\textbf{2}}
		First, it's necessary to prove that this set with the indicated operation is a group:
		\begin{itemize}
			\item [(i)] $x*(y*z)= \frac{x\frac{yz}{2}}{2}=\frac{xyz}{4}$ \\
			$(x*y)*z= \frac{\frac{xy}{2}z}{2}=\frac{xyz}{4}$ \\
			Therefore $*$ is an associative operation.
			\item [(ii)] $x*e=x$ then $\frac{xe}{2}=x \Rightarrow e=2$
			\item [(iii)] $x*x'=e$ then $\frac{xx'}{2}=e \Rightarrow x'=\frac{4}{x}$
			\item [(iv)] Finally, for this set with this operation to be an Abelian group, it's necessary that the operation be commutative, so \\
			$x*y =\frac{xy}{2}$ and $y*x=\frac{yx}{2}=\frac{xy}{2}$.\\
			Therefore $*$ is a commutative operation and with the set $\mathbb{R}$, they form an Abelian group.
		\end{itemize}
		\end{proof} 

	\subsubsection*{Problem B}
	Which of the following subsets of $\mathbb{R} \times \mathbb{R}$, with the indicated operation, is a group? Which is an Abelian group?
		\begin{itemize}
			\item [\textbf{1}] $(a,b)*(c,d)=(ad+bc,bd)$ on the set $\{(x,y) \in \mathbb{R} \times \mathbb{R}: y \neq 0\}$
			\item [\textbf{2}] $(a,b)*(c,d)=(ac,bc+d)$ on the set $\{(x,y) \in \mathbb{R} \times \mathbb{R}: x \neq 0\}$
		\end{itemize}
	
	\subsubsection*{Solutions}
	
		\begin{proof}{\textbf{1}}
		\begin{itemize}
			\item [(i)] Associative rule:\\
			$(e,f)*((a, b)*(c,d)) = (e,f)*(ad+bc,bd) = (ebd+f(ad+bc),fbd)=(ebd+fad+fbc,fbd)$ \\
			$((e,f)*(a,b))*(c,d) = (eb+fa,fb)*(c,d) = ((eb+fa)d+fbc,fbd)=(ebd+fad+fbc,fbd)$\\
			Therefore $*$ is an associative operation.
			\item [(ii)] Identity element: Solve $(a,b) *(e_1, e_2) = (a*b)$ so\\
			$(ae_2+be_1,be_2) = (a,b)$ then:
			\begin{equation*}
    			\begin{cases}
      				ae_2+be_1 = a\\
      				be_2=b
    			\end{cases}
			\end{equation*}
			Given that $b \neq 0$ then $e_2 = \frac{b}{b} = 1$\\
			replacing that result $a+be_1 = a$ then $e_1 = 0$
			\item [(iii)] Inverse: Solve $(a,b)*(a',b')=(e_1,e_2)$ so\\
			$(ab'+ba',bb') = (e_1,e_2) = (0,1)$ then:
			\begin{equation*}
    			\begin{cases}
      				ab'+ba' = 0\\
      				bb'=1
    			\end{cases}
			\end{equation*}
			Given that $b \neq 0$ then $b' = \frac{1}{b}$ \\
			and replacing that result $a\frac{1}{b} + ba'=0$ then $ba' = -\frac{a}{b} \Rightarrow a'=-\frac{a}{b^2}$
		\end{itemize}
		So $*$ with the set $\{(x,y) \in \mathbb{R} \times \mathbb{R}: y \neq 0\}$ is a group.\\
		Let's see if it's an Abelian group.
			\begin{itemize}
			\item [(iv)] Commutative rule:\\
			$(a,b)*(c,d) = (ad+bc,bd)$ and \\
			$(c,d)*(a,b) = (cb+da,bd)=(ad+bc,bd)$\\
			Therefore $*$ is a commutative rule. 
			\end{itemize}
		And so the operation $*$ with the set $\{(x,y) \in \mathbb{R} \times \mathbb{R}: y \neq 0\}$ is an Abelian group too.
		\end{proof}

		\begin{proof}{\textbf{2}}
		\begin{itemize}
			\item [(i)] Associative rule:\\
			$(e,f)*((a, b)*(c,d)) = (e,f)*(ac,bc+d) = (eac,fac+bc+d)$ and \\
			$((e,f)*(a,b))*(c,d) = (ea,fa+b)*(c,d) = (eac,(fa+b)c+d) = (eac,fac+bc+d)$\\
			Therefore $*$ is an associative operation.
			
			\item [(ii)] Identity element: Solve $(a,b) *(e_1, e_2) = (a*b)$ so\\
			$(ae_1, be_1+e_2) = (a,b)$ then:
			\begin{equation*}
    			\begin{cases}
      				ae_1 = a\\
      				be_1+e_2=b
    			\end{cases}
			\end{equation*}
			Given that $a \neq 0$ then $e_1 = \frac{a}{a} = 1$\\
			replacing that result $b+e_2 = b$ then $e_2 = 0$
			
			\item [(iii)] Inverse: Solve $(a,b)*(a',b')=(e_1,e_2)$ so\\
			$(aa',ba'+b') = (e_1,e_2) = (1,0)$ then:
			\begin{equation*}
    			\begin{cases}
      				aa' = 1\\
      				ba'+b'=0
    			\end{cases}
			\end{equation*}
			Given that $a \neq 0$ then $a' = \frac{1}{a}$ \\
			and replacing that result $b\frac{1}{a} + b'=0$ then $b' = -\frac{a}{b}$
		\end{itemize}
		So $*$ with the set $\{(x,y) \in \mathbb{R} \times \mathbb{R}: x \neq 0\}$ is a group.\\
		Let's see if it's an Abelian group.
			\begin{itemize}
			\item [(iv)] The operation $*$ is not a commutative rule, here a counter example is provided, let $(a,b)=(-1,0)$ and $(c,d)=(2,2)$ then:\\
			$(a,b)*(c,d) = ((-1) \cdot 2, 0 \cdot 2 + 2) = (-2,2)$ and \\
			$(c,d)*(a,b) = (2 \cdot (-1), 2 \cdot (-1) + 0)=(-2,-2)$\\ 
			\end{itemize}
		And so the operation $*$ with the set $\{(x,y) \in \mathbb{R} \times \mathbb{R}: x \neq 0\}$ is not an Abelian group.
		\end{proof}

	\subsubsection*{Problem C}
	If $D$ is a set, then the \textit{power set} of $D$ is the set $P_D$ of all the subsets of $D$. That is,
	\begin{equation*}
    	P_D = \{A:A \subseteq D\}
	\end{equation*}
	The operation $+$ is to be regarded as an operation on $P_D$
	\begin{itemize}
		\item [\textbf{1}] Prove that there is an identity element with respect to the operation $+$.
		\item [\textbf{2}] Prove every subset $A$ of $D$ has an inverse with respect to $+$.
		\item [\textbf{3}] Let $D$ be the three-element set $D=\{a,b,c\}$. List the elements of $P_D$. (For example, one element is $\{a\}$, another is $\{a,b\}$, and so on. Do not forget the empty set and the whole set $D$) Then write the operation table for $\langle P_D, +\rangle$.
	\end{itemize}
	\subsubsection*{Solutions}
		\begin{proof}{\textbf{1}}
			Identity element: Solve $A+e = A$ where $e$ is the identity element.\\
			Let $\{\}$ (the empty set) be the identity element, from the definition of the $+$ operation $A+B = A-B \cup B-A$ we have that\\
			$A+\{\} = A - \{\} \cup \{\} - A = A \cup A = A$.\\
			Therefore the empty set, $\{\}$ is the identity element.  		
		\end{proof}
		\begin{proof}{\textbf{2}}
			Inverse element: Solve $A+A'=\{\}$ since all the elements in $A$ are in $A$ $\Rightarrow A+A = \{\}$.\\
			Therefore $A' = A$ is the inverse element.
		\end{proof}
		\begin{proof}{\textbf{3}}
			The elements of $P_D$ are $\{\}, \{a\}, \{b\}, \{c\}, \{a,b\}, \{a,c\}, \{b,c\}, \{a,b,c\}$\\
			The operation table:\\
			\\
			\begin{adjustbox}{max width=\textwidth,center}
			\begin{tabular}{l|llllllll}
				$+$ & $\{\}$ & $\{a\}$ & $\{b\}$ & $\{c\}$ & $\{a,b\}$ & $\{a,c\}$ & $\{b,c\}$ & $\{a,b,c\}$ \\ \hline
				$\{\}$ & $\{\}$ & $\{a\}$ & $\{b\}$ & $\{c\}$ & $\{a,b\}$ & $\{a,c\}$ & $\{b,c\}$ & $\{a,b,c\}$ \\ 
				$\{a\}$ & $\{a\}$ & $\{\}$ & $\{a,b\}$ & $\{a,c\}$ & $\{b\}$ & $\{c\}$ & $\{a,b,c\}$ & $\{b,c\}$ \\
				$\{b\}$ & $\{b\}$ & $\{a,b\}$ & $\{\}$ & $\{b,c\}$ & $\{a\}$ & $\{a,b,c\}$ & $\{c\}$ & $\{a,c\}$ \\
				$\{c\}$ & $\{c\}$ & $\{a,c\}$ & $\{b,c\}$ & $\{\}$ & $\{a,b,c\}$ & $\{a\}$ & $\{b\}$ & $\{a,b\}$ \\
				$\{a,b\}$ & $\{a,b\}$ & $\{b\}$ & $\{a\}$ & $\{a,b,c\}$ & $\{\}$ & $\{b,c\}$ & $\{a,c\}$ & $\{c\}$ \\
				$\{a,c\}$ & $\{a,c\}$ & $\{c\}$ & $\{a,b,c\}$ & $\{a\}$ & $\{b,c\}$ & $\{\}$ & $\{a,b\}$ & $\{b\}$ \\
				$\{b,c\}$ & $\{b,c\}$ & $\{a,b,c\}$ & $\{c\}$ & $\{b\}$ & $\{a,c\}$ & $\{a,b\}$ & $\{\}$ & $\{a\}$ \\
				$\{a,b,c\}$ & $\{a,b,c\}$ & $\{b,c\}$ & $\{a,c\}$ & $\{a,b\}$ & $\{c\}$ & $\{b\}$ & $\{a\}$ & $\{\}$ \\
			\end{tabular}
			\end{adjustbox}
			\\\\
		\end{proof}
	
	
	
\end{document}
