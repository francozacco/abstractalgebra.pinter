\documentclass[11pt]{article}
\usepackage{amssymb}
\usepackage{amsthm}
\usepackage{enumitem}
\usepackage{amsmath}
\usepackage{bm}

\title{\textbf{Solved Abstract Algebra - Pinter}}
\author{Franco Zacco}
\date{}

\addtolength{\topmargin}{-3cm}
\addtolength{\textheight}{3cm}

\begin{document}

\maketitle
\thispagestyle{empty}

\section*{Chapter 3 - The definition of groups}

	\subsubsection*{Problem A}

		Prove that each of the following sets, with the indicated operation, is an Abelian group
		\begin{itemize}
			\item [\textbf{1}] $x * y = x+y+k$ ($k$ a fixed constant), on the set $\mathbb{R}$.
			\item [\textbf{2}] $x*y=\frac{xy}{2}$, on the set $\{ x \in \mathbb{R}: x \neq 0\}$
		\end{itemize}

	\subsubsection*{Solutions}

		\begin{proof}{\textbf{1}}
		First, it's necessary to prove that this set with the indicated operation is a group:
		\begin{itemize}
			\item [(i)] Let $l$ be a fixed constant then\\
			$x*(y*z)=x+(y+z+k)+l = x+y+z+k+l$ \\
			$(x*y)*z=(x+y+k)+z+l = x+y+k+z+l=x+y+z+k+l$ \\
			Therefore $*$ is an associative operation.
			\item [(ii)] $x*e=x$ then $x+e+k=x \Rightarrow e=-k$
			\item [(iii)] $x*x'=e$ then $x+x'+k=e \Rightarrow x'=-x-2k$
			\item [(iv)] Finally, for this set with this operation to be an Abelian group, it's necessary that the operation be commutative, so \\
			$x*y =x+y+k$ and $y*x=y+x+k=x+y+k$.\\
			Therefore $*$ is a commutative operation and with the set $\mathbb{R}$, they form an Abelian group.
		\end{itemize}
		\end{proof}

		\begin{proof}{\textbf{2}}
		First, it's necessary to prove that this set with the indicated operation is a group:
		\begin{itemize}
			\item [(i)] $x*(y*z)= \frac{x\frac{yz}{2}}{2}=\frac{xyz}{4}$ \\
			$(x*y)*z= \frac{\frac{xy}{2}z}{2}=\frac{xyz}{4}$ \\
			Therefore $*$ is an associative operation.
			\item [(ii)] $x*e=x$ then $\frac{xe}{2}=x \Rightarrow e=2$
			\item [(iii)] $x*x'=e$ then $\frac{xx'}{2}=e \Rightarrow x'=\frac{4}{x}$
			\item [(iv)] Finally, for this set with this operation to be an Abelian group, it's necessary that the operation be commutative, so \\
			$x*y =\frac{xy}{2}$ and $y*x=\frac{yx}{2}=\frac{xy}{2}$.\\
			Therefore $*$ is a commutative operation and with the set $\mathbb{R}$, they form an Abelian group.
		\end{itemize}
		\end{proof} 

\cleardoublepage 
	\subsubsection*{Problem B}

\end{document}
