\documentclass[11pt]{article}
\usepackage{amssymb}
\usepackage{amsthm}
\usepackage{enumitem}
\usepackage{amsmath}
\usepackage{bm}
\usepackage{adjustbox}

\title{\textbf{Solved Abstract Algebra - Pinter}}
\author{Franco Zacco}
\date{}

\addtolength{\topmargin}{-3cm}
\addtolength{\textheight}{3cm}

\begin{document}

\maketitle
\thispagestyle{empty}

\section*{Chapter 4 - Elementary properties of groups}

	\subsubsection*{Problem A}
	Solve:
	\begin{itemize}
		\item [\textbf{1}] $axb = c$
		\item [\textbf{2}] $x^2b = xa^{-1}c$
		\item [\textbf{4}] $ax^2=b$ and $x^3=e$
		\item [\textbf{5}] $x^2=a^2$ and $x^5=e$
	\end{itemize}
	\subsubsection*{Solutions}
		\begin{proof}{\textbf{1}}
			\begin{gather*}
				axb=c \\
				a^{-1}axb=a^{-1}c \\
				exb=a^{-1}c \\
				xb=a^{-1}c \\
				xbb^{-1}=a^{-1}cb^{-1} \\
				xe=a^{-1}cb^{-1} \\
				x=a^{-1}cb^{-1}
			\end{gather*}
		\end{proof}
		\begin{proof}{\textbf{2}}
			\begin{gather*}
				x^2b=xa^{-1}c \\
				x^{-1}x^2b=x^{-1}xa^{-1}c \\
				exb=ea^{-1}c \\
				xb=a^{-1}c \\
				xbb^{-1}=a^{-1}cb^{-1} \\
				xe=a^{-1}cb^{-1} \\
				x=a^{-1}cb^{-1}
			\end{gather*}
		\end{proof}
		\begin{proof}{\textbf{4}}
			\begin{gather*}
				ax^2=b \\
				a^{-1}ax^2=a^{-1}b \\
				ex^2=a^{-1}b \\
				x^2=a^{-1}b \\
				\text{multiplying by }x \text{ both sides}\\
				x^3=xa^{-1}b \\
				\text{and since } x^3=e \text{ then}\\
				xa^{-1}b=e \\
				xa^{-1}bb^{-1}=eb^{-1} \\
				xa^{-1}e=b^{-1} \\
				xa^{-1}=b^{-1} \\
				xa^{-1}a=b^{-1}a \\
				xe=b^{-1}a \\
				x=b^{-1}a			
			\end{gather*}
		\end{proof}
		\begin{proof}{\textbf{5}}
			\begin{gather*}
				x^5=e \\
				xx^2x^2=e \\
				\text{and since } x^2=a^2 \text{ then}\\
				xa^2a^2=e \\
				xa^4=e \\
				\text{by multiplying both sides with }(a^{-1})^4=a^{-4} \\
				xe=a^{-4} \\
				x=a^{-4}
			\end{gather*}
		\end{proof}
\cleardoublepage
	\subsubsection*{Problem B}
		For each of the following rules, either prove that it is true in every group $G$, or give a counterexample to show that it is false in some groups.
	\subsubsection*{Solutions}
		In all the counterexamples I'm using the matrices group $G = \{I,A,B,C,D,K\}$ with the following table.\\\\
		\begin{adjustbox}{max width=\textwidth,center}
		\begin{tabular}{l|llllll}
			$*$ & $I$ & $A$ & $B$ & $C$ & $D$ & $K$ \\ \hline
			$I$ & $I$ & $A$ & $B$ & $C$ & $D$ & $K$ \\
			$A$ & $A$ & $I$ & $C$ & $B$ & $K$ & $D$ \\
			$B$ & $B$ & $K$ & $D$ & $A$ & $I$ & $C$ \\
			$C$ & $C$ & $D$ & $K$ & $I$ & $A$ & $B$ \\
			$D$ & $D$ & $C$ & $I$ & $K$ & $B$ & $A$ \\
			$K$ & $K$ & $B$ & $A$ & $D$ & $C$ & $I$ \\
		\end{tabular}
		\end{adjustbox}
		\begin{proof}{\textbf{1}}\\
			Counterexample: Let $x=A$ and $A^2=I$ but $A \neq I$
		\end{proof}
		\begin{proof}{\textbf{2}}\\
			Counterexample: $A^2=I=C^2$ but $A \neq C$
		\end{proof}
		\begin{proof}{\textbf{3}}\\
			Counterexample: $(AB)^2=C^2=I$ on the other hand $A^2=I$ and $B^2=D$ then $A^2B^2=ID=D$ but $D \neq I$
		\end{proof}
		\begin{proof}{\textbf{4}}\\
			$x^2=x$ then $x^2x^{-1}=xx^{-1}$ so $x=e$
		\end{proof}
		\begin{proof}{\textbf{5}}\\
			Counterexample: For an $x=A$ there is no $y$ in $G$ such as $y^2=A$. This counterexample is easy to see in the operation table.
		\end{proof}
		\begin{proof}{\textbf{6}}\\
			$xz=y \Rightarrow x^{-1}xz=x^{-1}y \Rightarrow z=x^{-1}y$ for this to hold, the element $x^{-1}$ must exists and by the definition of groups, every group must have an inverse element for an element $x$, so $x^{-1} \in G$ and we also know by definition that $y \in G$, therefore $z=x^{-1}y \in G$
		\end{proof}
\cleardoublepage
	\subsubsection*{Solutions - Problem C}
		\begin{proof}{\textbf{1}}
			$a^{-1}b^{-1}=(ab)^{-1}=(ba)^{-1}=b^{-1}a^{-1}$
		\end{proof}
		\begin{proof}{\textbf{2}}
			Since $ba=ab$ multiplying on the left by $b^{-1}$ we get $a=b^{-1}ab$ then by replacing in $ab^{-1}$ we get that $ab^{-1}=(b^{-1}ab)b^{-1}=b^{-1}a$
		\end{proof}
		\begin{proof}{\textbf{3}}
			$a(ab)=a(ba)=(ab)a$
		\end{proof}
		\begin{proof}{\textbf{4}}
			$ab=ba$ multiplying on the left by $a$ we get $a^2b=(ab)a$ and then by multiplying on the right by $b$ we get $a^2b^2=(ab)(ab)=(ba)(ba)=b(ab)a=b(ba)a=b^2a^2$
		\end{proof}
		\begin{proof}{\textbf{5}}
			$(xax^{-1})(xbx^{-1})=xa(x^{-1}x)bx^{-1}=xabx^{-1}=xbax^{-1}=xbeax^{-1}=xbx^{-1}xax^{-1}=(xbx^{-1})(xax^{-1})$
		\end{proof}
		\begin{proof}{\textbf{6}}\\
			$ba=ab $ multiplying on the right by $a^{-1}$ then $ baa^{-1}=aba^{-1}$ so $b=aba^{-1}$\\
			$b=aba^{-1}$ multiplying on the right by $a$ then $ba=aba^{-1}a=ab$
		\end{proof}
		\begin{proof}{\textbf{7}}\\
			$ba=ab$ multiplying on the right by $a^{-1}$ we get $baa^{-1}=aba^{-1}$ then $b=aba^{-1}$ and multiplying on the right by $b^{-1}$ we get that $bb^{-1}=aba^{-1}b^{-1}$ then $e=aba^{-1}b^{-1}$\\
			$aba^{-1}b^{-1}=e$ multiplying on the right by $b$ we get that $aba^{-1}b^{-1}b=eb$ so $aba^{-1}=b$ multiplying on the right by $a$ then $aba^{-1}a=ba$ therefore $ab=ba$
		\end{proof}
	\subsubsection*{Solutions - Problem D}
		\begin{proof}{\textbf{1}}
			If $ab=e$ because of the Theorem 2, $b=a^{-1}$ multiplying on the right by $a$ then $ba=a^{-1}a$ therefore $ba=e$
		\end{proof}
		\begin{proof}{\textbf{2}}\\
			Since $abc=e$ because of the Theorem 2, $ab=c^{-1}$ multiplying on the left by $c$ we get that $cab=cc^{-1}$ therefore $cab=e$\\
			Since $abc=e$ because of the Theorem 2, $bc=a^{-1}$ multiplying on the right by $a$ we get that $bca=a^{-1}a$ therefore $bca=e$
		\end{proof}
		\begin{proof}{\textbf{4}}
			If $xay=a^{-1}$ multiplying on the right by $a$ we get $xaya=a^{-1}a=e$ because of the Theorem 2, $aya=x^{-1}$ multiplying on the right by $x$ we get that $ayax=x^{-1}x=e$ and now multiplying on the left by $a^{-1}$ we get $a^{-1}ayax=a^{-1}e$ therefore $yax=a^{-1}$
		\end{proof}
		\begin{proof}{\textbf{8}}\\
			$a=a^{-1}$ multiplying on the right by $a$ we get $aa=a^{-1}a$ therefore $a^2=e$\\
			$a^2=aa=e$ multiplying on the right by $a^{-1}$ we get that $aaa^{-1}=ea^{-1}$ therefore $a=a^{-1}$
		\end{proof}

	\subsubsection*{Solutions - Problem E}		
		\begin{proof}{\textbf{1}}
			$S$ can be written as $S = \bigcup_{i=1}^n \{x_i, x_i^{-1}\}$ by pairing each element with its inverse, therefore there must be an even set of elements in $S$.
		\end{proof}
		\begin{proof}{\textbf{2}}
			By definition $G$ can be written as $G = \{x \in G: x=x^{-1}\} \cup S$ where $S = \{x \in G: x \neq x^{-1}\}$.\\
			If $G$ has an even set of numbers and it's known that $S$ has an even set of numbers, then for this to hold $\{x \in G: x=x^{-1}\}$ must have an even set of numbers.\\
			On the other hand if $G$ has an odd set of numbers then for this to hold $\{x \in G: x=x^{-1}\}$ must have an odd set of numbers.
		\end{proof}
		\begin{proof}{\textbf{3}}
			Since the order of $G$ is even then the order of $\{x \in G: x=x^{-1}\}$ is even, on the other hand, since $e=e^{-1}$ is not in $S$ then the order of $\{x \in G: x \neq e \land x=x^{-1}\}$ is odd, therefore it has at least one element.
		\end{proof}	
		\begin{proof}{\textbf{4}}
			By definition $G = \{e\} \cup \{x \in G:x \neq e $ and $ x=x^{-1}\} \cup S$ since $G$ is Abelian we can solve $(a_1a_2...a_n)^2$ separately and then multiply each result. Let $T=\{x \in G:x \neq e $ and $ x=x^{-1}\}$\\
			The set $T$ can be rearranged as \\$\displaystyle (a_1a_2...a_m)^2=\prod_{i=1}^{m} a_ia_i=e$ where $m$ is the order of $T$.\\
			The set $S$ can be rearranged so each element is multiplied by its inverse, then $\displaystyle (a_1a_2...a_k)^2=\prod_{i=1}^{k} a_ia_i^{-1}=e$ where $k$ is the order of $S$.\\
			Finally we see that $(a_1a_2...a_n)^2=(a_1a_2...a_m)^2(a_1a_2...a_k)^2=ee=e$´
		\end{proof}
		\begin{proof}{\textbf{5}}
			If there is no element $x \neq e$ in $G$ such that $x=x^{-1}$ then $G=S$ then the product can be rearranged as $\displaystyle a_1a_2...a_n=\prod_{i=1}^{n/2} a_ia_i^{-1}=e$.
		\end{proof}
		\begin{proof}{\textbf{6}}
			If there is exactly one $x \neq e$ in $G$ such that $x=x^{-1}$ then\\ $\displaystyle G=S \cup \{x\}$ and since $G$ is an Abelian group we can solve separately for each set, so for $S$ we saw that $a_1a_2...a_m=e$, where $m$ is the order of $S$, then $a_1a_2...a_n=a_1a_2...a_mx=ex=x$  
		\end{proof}
\cleardoublepage
	\subsubsection*{Solutions - Problem F}
		\begin{proof}{\textbf{1}}
			\begin{itemize}
				\item[(a)] If $a^2=a$ then multiplying on the right by $a^{-1}$ we get $aaa^{-1}=aa^{-1}$ therefore $a=e$.
				\item[(b)] If $ab=a$ then multiplying on the left by $a^{-1}$ we get $a^{-1}ab=a^{-1}a$ therefore $b=e$
				\item[(c)] If $ab=b$ then multiplying on the right by $b^{-1}$ we get $abb^{-1}=bb^{-1}$ therefore $a=e$
			\end{itemize}
		\end{proof}
		\begin{proof}{\textbf{2}}
			Let $a, x, y_1, y_2$ be different elements of $G$, respecting the following table\\
			\begin{adjustbox}{max width=\textwidth,center}
			\begin{tabular}{l|lllll}
				  & $...$ & $y_1$ & $...$ & $y_2$ & $...$ \\ \hline
				$\vdots$ &  & $\vdots$ &  & $\vdots$ & \\
				$a$ & $...$ & $x$ & $...$ & $x$ &  \\
			\end{tabular}
			\end{adjustbox}
		\\\\So $ay_1=x$ and $ay_2=x$ then $ay_1=ay_2$ multiplying on the left by $a^{-1}$ we get $a^{-1}ay_1=a^{-1}ay_2$ then $y_1=y_2$ which is a contradiction then $x$ cannot be twice in a row.\\
		Now, let $x$ not being in a row then since the row must have $n$ elements, where $n$ is the order of $G$, since $x$ is not in the row then we have $n-1$ elements to complete the row, so another element $y \in G$ must be twice in the row, but we saw that it's not possible. Therefore $x$ must be once in the row.\\
		Therefore every row of a group table must contain each element of the group exactly once.
		\end{proof}
		\begin{proof}{\textbf{3}}
			Since the first row and column are fixed because of the identity element, the only way of completing the table without repeating the values in each column is as follows\\\\
			\begin{adjustbox}{max width=\textwidth,center}
			\begin{tabular}{l|lll}
				  & $e$ & $a$ & $b$ \\ \hline
				$e$ & $e$ & $a$ & $b$ \\
				$a$ & $a$ & $b$ & $e$ \\
				$b$ & $b$ & $e$ & $a$ \\
			\end{tabular}
			\end{adjustbox}
		\end{proof}
\cleardoublepage
		\begin{proof}{\textbf{4}}
			As we noticed before the first row and column are fixed because of the identity element, also because of the added property that $xx=e$ for every $x \in G$ then the diagonal of the table is filled with the $e$ element.\\
			For the row $a$ column $b$ we also saw that if $ab=b$ then $a=e$ but $a \neq e$ so must be that $ab=c$. The same can be applied to fill the remaining places.\\\\
			\begin{adjustbox}{max width=\textwidth,center}
			\begin{tabular}{l|llll}
				  & $e$ & $a$ & $b$ & $c$ \\ \hline
				$e$ & $e$ & $a$ & $b$ & $c$\\
				$a$ & $a$ & $e$ & $c$ & $b$ \\
				$b$ & $b$ & $c$ & $e$ & $a$ \\
				$c$ & $c$ & $b$ & $a$ & $e$ \\
			\end{tabular}
			\end{adjustbox}
		\end{proof}
		\begin{proof}{\textbf{5}}
			In this case since we have two added properties $aa=e$ and $bb \neq e$ then in row and column $a$ we must place $e$, the other $e$ element must be placed in the last column $c$, so every element appear only once in each row/column and the property $bb \neq e$ holds. The rest of the table can be filled by avoiding repeated elements in each row/column.\\\\
			\begin{adjustbox}{max width=\textwidth,center}
			\begin{tabular}{l|llll}
				    & $e$ & $a$ & $b$ & $c$ \\ \hline
				$e$ & $e$ & $a$ & $b$ & $c$\\
				$a$ & $a$ & $e$ & $c$ & $b$ \\
				$b$ & $b$ & $c$ & $a$ & $e$ \\
				$c$ & $c$ & $b$ & $e$ & $a$ \\
			\end{tabular}
			\end{adjustbox}
		\end{proof}
		\begin{proof}{\textbf{6}}
			The order of $G$ is 4, which is an even number, so the set $\{x \in G: x=x^{-1}\}$ has an even order too, then there are two posibilities, the 4 elements are it's own inverse or 2 of the elements are it's own inverse, which are the two options shown before.
		\end{proof}

\cleardoublepage
	\subsubsection*{Solutions - Problem G}
		\begin{proof}{\textbf{1}} $G \times H$ is a group.
			\begin{itemize}
				\item[(G1)] Association.\\
				$(x_1,y_1)[(x_2,y_2)(x_3,y_3)]=(x_1,y_1)(x_2x_3,y_2y_3)=(x_1x_2x_3,y_1y_2y_3)$\\
				$[(x_1,y_1)(x_2,y_2)](x_3,y_3)=(x_1x_2,y_1y_2)(x_3,y_3)=(x_1x_2x_3,y_1y_2y_3)$\\
				\item[(G2)] Identity element.\\
				Let $(e_G, e_H)$ be the identity element, where $e_G$ is the identity element of $G$ and $e_H$ is the identity element of $H$ then\\
				$(x,y)(e_G,e_H)=(xe_G,ye_H)=(x,y)$
				\item[(G3)] Inverse element.\\
				Let $(x^{-1},y^{-1})$ be the inverse element of element $(x,y)$, where $x^{-1}$ is the inverse element of $x$ in $G$ and $y^{-1}$ is the inverse element of $y$ in $H$ then $(x,y)(x^{-1},y^{-1}) = (xx^{-1}, yy^{-1})=(e_G, e_H)$	
			\end{itemize}
		\end{proof}
		\begin{proof}{\textbf{2}}
			$\mathbb{Z}_2 \times \mathbb{Z}_3 = \{(0,0),(0,1),(0,2),(1,0),(1,1),(1,2)\}$\\\\
			\begin{adjustbox}{max width=\textwidth,center}
			\begin{tabular}{l|llllll}
				   +    & $(0,0)$ & $(0,1)$ & $(0,2)$ & $(1,0)$ & $(1,1)$ & $(1,2)$ \\ \hline
				$(0,0)$ & $(0,0)$ & $(0,1)$ & $(0,2)$ & $(1,0)$ & $(1,1)$ & $(1,2)$ \\
				$(0,1)$ & $(0,1)$ & $(0,2)$ & $(0,0)$ & $(1,1)$ & $(1,2)$ & $(1,0)$ \\
				$(0,2)$ & $(0,2)$ & $(0,0)$ & $(0,1)$ & $(1,2)$ & $(1,0)$ & $(1,1)$ \\
				$(1,0)$ & $(1,0)$ & $(1,1)$ & $(1,2)$ & $(0,0)$ & $(0,1)$ & $(0,2)$ \\
				$(1,1)$ & $(1,1)$ & $(1,2)$ & $(1,0)$ & $(0,1)$ & $(0,2)$ & $(0,0)$ \\
				$(1,2)$ & $(1,2)$ & $(1,0)$ & $(1,1)$ & $(0,2)$ & $(0,0)$ & $(0,1)$ \\
			\end{tabular}
			\end{adjustbox}
		\end{proof}
		\begin{proof}{\textbf{3}} Let $x_1,x_2 \in G$ and $y_1,y_2 \in H$, then
			\begin{align*}
				 (x_1, y_1)(x_2, y_2) & = (x_1x_2, y_1y_2) \\
					& = (x_2x_1, y_2y_1) \text{ since } G \text{ and } H \text{ are Abelian} \\
					& = (x_2,y_2)(x_1,y_1)
			\end{align*}
		\end{proof}
		\begin{proof}{\textbf{4}} Let $x \in G$ and $y \in H$ then
			\begin{align*}
				(x,y)(x,y) & = (xx,yy) \\
				 & = (e_G, e_H) \text{ since in } G \text{ and } H \text{ all elements are  its own inverses}			 
			\end{align*}
			Since $(e_G, e_H)$ is the inverse element of $G \times H$, then each element in $G \times H$ is its own inverse.
		\end{proof}
\cleardoublepage
	\subsubsection*{Solutions - Problem H}
		\begin{proof}{\textbf{1}}
			Let $A$ consist of all the positive integers $n$ for which $(bab^{-1})^n = ba^{n}b^{-1}$ is true.\\
			Then 1 is in $A$ because
				\begin{align*}
					(bab^{-1})^1 = bab^{-1} = ba^1b^{-1}
				\end{align*}
			Now let $k$ be any positive integer in $A$, we must show that $k+1$ is also in $A$. Saying that $k$ is in $A$ means that
				\begin{align*}
					(bab^{-1})^k = ba^{k}b^{-1}
				\end{align*}
			Multiplying by $bab^{-1}$ both sides of the equation we have that
				\begin{align*}
					(bab^{-1})(bab^{-1})^k & = (bab^{-1})(ba^{k}b^{-1})\\
					(bab^{-1})^{k+1} & = ba(b^{-1}b)a^{k}b^{-1} \\
					& = baa^{k}b^{-1} \\
					& = ba^{k+1}b^{-1}
				\end{align*}
			From this last equation we see that $k+1 \in A$.\\
			So by the principle of mathematical induction, all positive integers are in $A$. Therefore $(bab^{-1})^n = ba^{n}b^{-1}$ is true for every positive integer.
		\end{proof}
		\begin{proof}{\textbf{4}}
			If $a^{3}=e$ then multiplying on the right by $a^{-1}$ we get that \\ $a^{3}a^{-1} =a^{-1}$ then $a^2=a^{-1}$.\\
			Multiplying on the right by $a^{-1}$ again we get that $a^{2}a^{-1} = a^{-1}a^{-1}$ then $a=(a^{-1})^{2}$. Therefore $a$ has a square root.
		\end{proof}
		\begin{proof}{\textbf{5}}
			If $a^{2}=e$ then by multipying on the right by $a$ we get that $a^{3} = a$. Therefore $a$ has a cube root.
		\end{proof}
		\begin{proof}{\textbf{6}}
			If $a^{-1}$ has a cube root then
			\begin{align*}
				a^{-1} & = z^3 \\
				aa^{-1} & = az^{3} \text{ multiplying by }a \text{ both sides }\\
				e & = az^{3} \\ 
				z^{-1} & = a z^{3}z^{-1} \text{ multiplying by } z^{-1} \text{ both sides }\\
				z^{-1} & = az^2 \\
				z^{-1}z^{-1} & = az^2z^{-1} \text{ multiplying by } z^{-1} \text{ both sides }\\
				(z^{-1})^{2} & = az \\
				(z^{-1})^{2}z^{-1} & = azz^{-1} \text{ multiplying by } z^{-1} \text{ both sides }\\
				(z^{-1})^{3} & = a
			\end{align*}
			Therefore $a$ has a cube root too.
		\end{proof}
\end{document}


































