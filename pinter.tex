\documentclass[11pt]{article}
\usepackage{amssymb}
\usepackage{amsthm}
\usepackage{enumitem}
\usepackage{amsmath}
\usepackage{bm}

\title{\textbf{Solved Abstract Algebra - Pinter}}
\author{Franco Zacco}
\date{}

\addtolength{\topmargin}{-3cm}
\addtolength{\textheight}{3cm}

\begin{document}

\maketitle
\thispagestyle{empty}

\section*{Chapter 2 - Operations}

\subsubsection*{Problem A}

Which of the following rules are operations on the indicated set?
\begin{itemize}
\item [(1)] $a * b = \sqrt{\mathopen|ab\mathclose|}$ on the set $\mathbb{Q}$
\item [(3)] $a * b$ is a root of the equation $x^2 - a^2b^2 = 0$, on the set $\mathbb{R}$
\item [(5)] Subtraction, on the set $\{n \in \mathbb{Z}: n \geq 0\}$
\item [(6)] $a * b = \mathopen|a - b\mathclose|$ on the set $\{n \in \mathbb{Z}: n \geq 0\}$
\end{itemize}

\subsubsection*{Solutions}

\begin{proof}{(1)}
This is not an operation on $\mathbb{Q}$ since we can find $a$ and $b$ which are in $\mathbb{Q}$ such that $\sqrt{\mathopen|ab\mathclose|}$ is not in $\mathbb{Q}$. For example $a=1$ and $b=2$ then $\sqrt{\mathopen|ab\mathclose|} = \sqrt{\mathopen|2\mathclose|} = \sqrt{2}$. Thus, $\mathbb{Q}$ is not closed under $*$
\end{proof} 
\begin{proof}{(3)}
Ths is not an operation on $\mathbb{R}$ since $a * b$ is not uniquely defined for any $a,b\in\mathbb{R},a\neq0,b\neq0$. If $a\neq0,b\neq0$, then the equation $x^2 - a^2b^2 = 0$ has two roots, namely $x=a*b=\pm ab$
\end{proof} 
\begin{proof}{(5)}
This is not an operation on $\{n \in \mathbb{Z}: n \geq 0\}$ since we can find two integers, $n_1$ and $n_2$ that are in this set so $n_1 - n_2 \not\in \{n \in \mathbb{Z}: n \geq 0\}$. For example $0 - 1 \not\in \{n \in \mathbb{Z}: n \geq 0\}$.
\end{proof}
\begin{proof}{(6)}
This is an operation on $\{n \in \mathbb{Z}: n \geq 0\}$, because \\
if $a > b$ then $a-b> 0$ and $\mathopen|a-b\mathclose|> 0$ so $\mathopen|a-b\mathclose| \in\{n \in \mathbb{Z}: n \geq 0\}$ \\
if $a < b$ then $a-b < 0$ and $\mathopen|a-b\mathclose|> 0$ so $\mathopen|a-b\mathclose| \in\{n \in \mathbb{Z}: n \geq 0\}$ \\
and finally if $a=b$ then $a-b=0$ and $\mathopen|a-b\mathclose|= 0$ so $\mathopen|a-b\mathclose| \in\{n \in \mathbb{Z}: n \geq 0\}$
\end{proof}

\cleardoublepage 
\subsubsection*{Problem B}

Each of the following is an operation * on $\mathbb{R}$. Indicate whether or not
\begin{itemize}
\item [(i)] it is commutative,
\item [(ii)] it is associative,
\item [(iii)] $\mathbb{R}$ has an identity element with respect to $*$,
\item [(iv)] every $x \in \mathbb{R}$ has an inverse with respect to $*$.
\end{itemize}
Selected problems:
\begin{itemize}
\item [3.] $x*y=\mathopen|x+y\mathclose|$
\item [4.] $x*y=\mathopen|x-y\mathclose|$
\item [5.] $x * y = xy+1$
\end{itemize}

\subsubsection*{Solutions}
\begin{proof}{3.}
\begin{itemize}
\item [(i)] $a * b = \mathopen|a+b\mathclose|$ and $b * a = \mathopen|b+a\mathclose| = \mathopen|a+b\mathclose|$ thus $*$ is commutative.
\item [(ii)] $a*b$ is not associative, an example is given below: \\
$-2 * (1 * 2) = -2 * \mathopen|1+2\mathclose| = \mathopen|-2 + \mathopen|1+2\mathclose|\mathclose|= \mathopen|-2+3\mathclose|= 1$\\
$(-2 * 1) * 2 = \mathopen|-2+1\mathclose| * 2 = \mathopen|\mathopen|-2 + 1\mathclose| +2\mathclose|= \mathopen|1+2\mathclose|= 3$
\item [(iii)] We need to solve this equation $x*e=x$ to determine $e$ then we do $\mathopen|x+e\mathclose| =x$ so $x+e = x \Rightarrow e = 0$ or $-(x+e) = x \Rightarrow e = -2x$. For the first one if $x=-2$ then $\mathopen|-2+0\mathclose| \neq -2$ so $e=0$ is not an identity element and the other solution is dropped instantly given that $e$ cannot depend on $x$. Therefore $*$ doesn't have an identity element.
\item [(iv)] Since $*$ doesn't have an identity element, thus $*$ doesn't have an inverse. 
\end{itemize}
\end{proof}
\begin{proof}{4.}
\begin{itemize}
\item [(i)] if $a>b$ then $a * b = \mathopen|a-b\mathclose| = a-b$ and $b * a = \mathopen|b-a\mathclose| = -(b-a)=a-b$\\
if $a=b$ then $a * b = \mathopen|a-b\mathclose| = 0$ and $b * a = \mathopen|b-a\mathclose| = 0$\\
if $a<b$ then $a * b = \mathopen|a-b\mathclose| = -(a-b) = b-a$ and $b * a = \mathopen|b-a\mathclose| = b-a$\\
Therefore, * is commutative.
\item [(ii)] $a*b$ is not associative, a counter example is given below: \\
$-1 * (0 * 2) = -1 * \mathopen|0-2\mathclose| = \mathopen|-1 - \mathopen|0-2\mathclose|\mathclose|= \mathopen|-1-2\mathclose|= 3$\\
$(-1 * 0) * 2 = \mathopen|-1-0\mathclose| * 2 = \mathopen|\mathopen|-1-0\mathclose| -2\mathclose|= \mathopen|1-2\mathclose|= 1$
\item [(iii)] We need to solve this equation $x*e=x$ to determine $e$ then we do $\mathopen|x-e\mathclose| =x$ so $x-e = x \Rightarrow e = 0$ or $-(x-e) = x \Rightarrow e = 2x$. So if $x=-2$ then $\mathopen|-2-0\mathclose| \neq -2$ so $e=0$ is not an identity element of $*$ and the other solution for $e$ is dropped instantly given that $e$ cannot depend on $x$. Therefore $*$ doesn't have an identity element.
\item [(iv)] Since $*$ doesn't have an identity element, thus $*$ doesn't have an inverse. 
\end{itemize}
\end{proof}

\begin{proof}{5.}
\begin{itemize}
\item [(i)] Since $x*y = xy + 1$ and $y*x=yx+1=xy+1$ then $*$ is commutative.
\item [(ii)] $x*y$ is not associative, a counter example is given below: \\
$1 * (0 * 2) = 1\cdot(0\cdot2 +1) +1 = 2$\\
$(1 * 0) * 2 = (1\cdot0 +1)\cdot2 +1 = 3$
\item [(iii)] We need to solve this equation $x*e=x$ to determine $e$ then we do $xe+1=x$ so $e=(x-1)/x=1-(1/x)$, since $e$ cannot depend on $x$, $*$ doesn't have an identity element.
\item [(iv)] Since $*$ doesn't have an identity element, thus $*$ doesn't have an inverse. 
\end{itemize}
\end{proof}
\subsubsection*{Problem D}
An input sequence is a finite sequence of symbols from some alphabet $A$.\\*
The set of all sequences of symbols in the alphabet $A$ is denoted by $A^*$.
There is an operation on $A^*$ called $concatenation$: If $\bm{a}$ and $\bm{b}$ are in $A^*$, say $\bm{a} = a_1a_2 \ldots a_n$ and $\bm{b} = b_1b_2 \ldots b_n$, then
\begin{gather*}
\bm{ab} = a_1a_2 \ldots a_nb_1b_2 \ldots b_n
\end{gather*}
In other words, the sequence $\bm{ab}$ consists of two sequences $\bm{a}$ and $\bm{b}$ end to end.\\
The symbol $\lambda$ denotes the empty sequence
\begin{itemize}
\item [\textbf{1}] Prove that the operation defined above is associative.
\item [\textbf{2}] Explain why the operation is not commutative.
\item [\textbf{3}] Prove that there is an identity element for this operation.
\end{itemize}
\subsubsection*{Solutions}
\begin{itemize}
\item [\textbf{1}] $\bm{c}(\bm{a}\bm{b}) = \bm{c}(a_1a_2 \ldots a_nb_1b_2 \ldots b_n)=c_1c_2 \ldots c_na_1a_2 \ldots a_nb_1b_2 \ldots b_n$\\
$(\bm{c}\bm{a})\bm{b} = (c_1c_2 \ldots c_na_1a_2 \ldots a_n)\bm{b}=c_1c_2 \ldots c_na_1a_2 \ldots a_nb_1b_2 \ldots b_n$.\\
Therefore the $concatenation$ operation is associative.
\item [\textbf{2}] Suppose $\bm{a}=000$ and $\bm{b}=111$ then $\bm{ab}=000111$ but $\bm{ba}=111000$ so $\bm{ab} \neq \bm{ba}$. Thus the $concatenation$ operation is not commutative.
\item [\textbf{3}] $e=\lambda$ it's the identity element for the $concatenation$ operation, since $\bm{a} \lambda=\lambda \bm{a}=\bm{a}$    
\end{itemize}

\end{document}
